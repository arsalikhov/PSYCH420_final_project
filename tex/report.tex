\documentclass{article}

\usepackage{listings}
\usepackage{mathtools}
\usepackage{apacite}
\usepackage{blindtext}
\usepackage{titling}
\renewcommand\maketitlehooka{\null\mbox{}\vfill}
\renewcommand\maketitlehookd{\vfill\null}
\usepackage{hyperref}
\hypersetup{
    colorlinks=true,
    linkcolor=blue,
    filecolor=magenta,      
    urlcolor=cyan,
    pdftitle={Overleaf Example},
    pdfpagemode=FullScreen,
    }

\title{Use of Deep Spiking Neural Network for Object Recognition}
\author{Arslan Salikhov  \\
	\and 
	Erik Caceros \\
	\and
	Brandon Lam \\
	}
\date{\today}



\begin{document}

\begin{titlingpage}
\maketitle
\end{titlingpage}


\tableofcontents
\newpage


\begin{abstract}
Use of Deep Neural Network, commonly referred to as
\emph{deep learning} spiked in recent years and has been used
as a tool for impressive advancements in the field of 
\emph{Artificial Intelligence (AI)}
Spiking Neural Networks draw inspiration from the 
Purpose of this project is to demonstrate the capabilities of a 
Spiking Neural Network and compare it to a more conventional 
Object Recognition Deep Neural Network
\end{abstract}

\section{Introduction}

The dataset used for training, validating and testing the 
model was assembled by Microsoft and will be referred as
COCO throughout this report (\citeA{COCOdataset}).

\subsection{Imbalanced Threshold} 
\[
    f(x)= 
\begin{dcases}
    1 ,& \text{if } {V_{mem}}\geq V_{th,pos}(V_{th})\\
	-1 ,& \text{if } {V_{mem}}\geq V_{th,neg}(-\frac{1}{\alpha}V_{th})\\
    0,              & \text{otherwise, no firing}
\end{dcases}
\]

\newpage
\bibliographystyle{apacite}
\bibliography{report}

\end{document}